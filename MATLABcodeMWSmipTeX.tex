

%%%%%%%%%%%%%%%%%%%%%%%%%%%%%%%%%%%%%%%%%%%%%%%%




\documentclass[12pt]{article}
\usepackage{graphicx,bm,amsfonts,amsmath,setspace}
%\usepackage[nolists]{endfloat}
\usepackage{color,hyperref,verbatim,rotating,subfigure}
\usepackage{apacite}
%\usepackage{listings}
\usepackage{listings}
\usepackage{color}

\definecolor{dkgreen}{rgb}{0,0.6,0}
\definecolor{gray}{rgb}{0.5,0.5,0.5}
\definecolor{mauve}{rgb}{0.58,0,0.82}

\lstset{frame=tb,
  %language=Java,
  language=Matlab,
  aboveskip=3mm,
  belowskip=3mm,
  showstringspaces=false,
  columns=flexible,
  basicstyle={\small\ttfamily},
  numbers=none,
  numberstyle=\tiny\color{gray},
  keywordstyle=\color{blue},
  commentstyle=\color{dkgreen},
  stringstyle=\color{mauve},
  breaklines=true,
  breakatwhitespace=true,
  tabsize=3
}

%\renewcommand{\baselinestretch}{1.5}
\newcommand{\ve}[1]{\mbox{\boldmath $#1$}}
\newcommand{\diag}{\mathop{\rm diag}}
\def\argmax{\mathop{\rm arg\,max}}%
\setlength{\textheight}{9.8in}
\setlength{\textwidth}{6.53in}
\oddsidemargin=0mm
\evensidemargin=0mm
\topmargin=0mm
\hoffset=0mm
\voffset=-0.5in
\textheight=9.4in
%\setcounter{secnumdepth}{0}
% THEOREM Environments ---------------------------------------------------
 \newtheorem{thm}{{\bf Theorem}}
% \newtheorem{cor}[thm]{Corollary}
% \newtheorem{lem}[thm]{Lemma}
% \newtheorem{prop}[thm]{Proposition}
% \theoremstyle{definition}
% \newtheorem{defn}[thm]{Definition}
% \theoremstyle{remark}
% \newtheorem{rem}[thm]{Remark}
% \numberwithin{equation}{subsection}
%%% ----------------------------------------------------------------------
\bibliographystyle{apacite}
%\bibliographystyle{asa}

%%%%%%%%%%%%%%%%%%%%%%%%%%%%%%%%%%%%%%%%%%%%%%%%

\begin{document}

\section{Code MATLAB for MWS} \label{Sec:Contents}

\textbf{MaxScoreCompute.m}

\begin{lstlisting}

function [value,estimates,time,quality]=MaxScoreCompute()

%Main Program: Computes Max score by defining a MILP and calling milp.m
%tic
[X,y,w]=readXyw();
[X,mu,sigma]=standardizeX(X);
[c,A,b]=definecAb(X,y,w);
[lb,ub, Aeq, beq, n, best]=definelbub(X);
[x,score,feasible, time]=milp_cplex(c,A,b,Aeq,beq,lb,ub,n, best);
estimatesNorm=x
value=score
status=feasible
runtime=time
quality=status

estimatesRaw=denormalizeEstimates(estimatesNorm,mu,sigma)
estimates=estimatesRaw
%time=toc
% HILSquality=feasible;

end

\end{lstlisting}


\textbf{definecAb.m}

\begin{lstlisting}

function [c,A,b]=definecAb(X,y,w)

%Defines c,A,b for milp.m

n=size(X,1);
p=size(X,2);

c1=repmat(-1,1,n);
c2=repmat(0,1,p);
c=[c1 c2];

d=10;
%d=5;
for i=1:1:n
    M(i)=abs(X(i,1))+abs(X(i,2:end))*repmat(d,1,p-1)';
end
Abin=diag([M]);
for i=1:1:n
    for j=1:1:p
        Areal(i,j)=(1-2.*y(i)) * X(i,j) ;
    end
end
A=[Abin Areal];

b=M;

end

\end{lstlisting}



\textbf{definelbub.m}

\begin{lstlisting}

function [lb,ub, Aeq, beq, n, best]=definelbub(X)

%Defines lb,ub for milp.m
d=10;
%d=5;
n=size(X,1);
p=size(X,2);
lb1=repmat(0,1,n);
lb2=repmat(-d,1,p);
lb2(1)=1;
lb=[lb1 lb2];

ub1=repmat(1,1,n);
ub2=repmat(d,1,p);
ub2(1)=1;
ub=[ub1 ub2];

Aeq=[];
beq=[];

best=0;

end

\end{lstlisting}






\textbf{milp\_cplex.m}

\begin{lstlisting}

function [x,score,feasible, time]=milp_cplex(c,A,b,Aeq,beq,lb,ub,n, best);   
% Solves a mixed integer lp using gurobi 5.0.1 
% c: is objective function coefficients A: is constraint matrix   b: is constraint vector
% lb: lower bound  ub: upper bound  n: number of 0-1 variables
% best: is best solution so far
% Note this uses the MATLAB/Gurobi Interface documented at
% http://www.gurobi.com/documentation/5.0/reference-manual/node650
% Also, it assumes first n variables must be integer.
% The MIP equations of the maximum score estimator are available at
% Florios. K, Skouras, S. (2008) Exact computation of maximum weighted
% score estimators, Journal of Econometrics 146, 86-91.
% Written by Kostas Florios, July 20, 2012
%
% gurobi 5.0.1
% cd c:/Users/jones/gurobi500/win64/matlab
% gurobi_setup
%
% cd C:\gurobi501\win32\matlab
% gurobi_setup
%
% cplex 12.6
% addpath('C:\Program Files\IBM\ILOG\CPLEX_Studio_Preview126\cplex\matlab\x86_win32')
%
model.Aineq = sparse(A) ;
model.f = c ;
model.bineq = b ;
model.lb = lb ;
model.ub = ub ;
model.ctype = [repmat('B',size(b,2),1) ; repmat('C',size(c,2)-size(b,2),1)]' ;


opt = cplexoptimset('cplex') ;
opt.mip.display = 4 ;
opt.mip.interval = 1000 ;

opt.timelimit= 100 ;
opt.mip.tolerances.mipgap = 0.00 ;
opt.parallel = 1 ;
opt.threads = 4 ;
opt.mip.strategy.file = 3 ; 
opt.workmem = 1024 ;
opt.emphasis.mip = 3 ;

opt.exportmodel = 'D:\storage\research\matlab2011\milp-cplex-2015\myModel.lp' ;

model.options = opt;


[x,fval,exitflag,output] = cplexmilp(model) ;

  fprintf('Optimization returned status: %s\n', output.message);
  fprintf('Objective Value: %e\n', fval);
  fprintf('(Wall clock) Time elapsed (s): %e\n', output.time);
  fprintf('Decision variables: the last ones (after the 1.00) are betas\n');
  %disp(x)


x=x((size(b,2)+1):size(c,2))
score=-fval;
feasible=output.message;
time=output.time;
return;


\end{lstlisting}



\textbf{readXyw.m}

\begin{lstlisting}

function [X,y,w]=readXyw()

%Reads X,y,w of given max score problem

X=load('X_Horowitz.txt');
%X=[X(:,2) X(:,3) X(:,4)];
X=X(:,2:end);
y=load('y_Horowitz.txt');
y=[y(:,2)];
w=load('w_Horowitz.txt');
w=[w(:,2)];


end

\end{lstlisting}



\textbf{denormalizeEstimates.m}

\begin{lstlisting}

function [estimatesRaw]=denormalizeEstimates(estimatesNorm,mu,sigma);

%denormalized estimatesNorm obtained by Gurobi MIP to estimatesRaw, which
%are meaningful to the user

%quick and dirty implementation, based on GAMS and Fortran Analogues
p=size(estimatesNorm,1);
betaNorm=estimatesNorm;

for j=1:1:p
    betaRaw(j)=0;
    betaHelp(j)=0;
end

for j=1:1:p
    if (sigma(j) ~=0) 
        betaHelp(j)= betaNorm(j)./sigma(j);
    end
    if (sigma(j) ==0)
        for jj=1:1:p
            if (sigma(jj) ~=0)
                betaHelp(j)=betaHelp(j)-betaNorm(jj).*mu(jj)./sigma(jj) ;
            else
                jj0=jj;
            end
        end
        betaHelp(j)=betaHelp(j)+betaNorm(jj0);
    end
end

for j=1:1:p
   betaRaw(j)=betaHelp(j)./betaHelp(1); 
end

estimatesRaw=betaRaw;

end

\end{lstlisting}



\textbf{standardizeX.m}

\begin{lstlisting}

function [X,mu,sigma]=standardizeX(X)

%Standardizes X

mu=mean(X);
sigma=std(X);
testX=(X-repmat(mu,size(X,1),1)) ./ repmat(sigma,size(X,1),1);
p=size(X,2);
for j=1:1:p
if isnan(testX(:,j))
   X(:,j) = X(:,j);
else
    X(:,j) = testX(:,j);
end

end

\end{lstlisting}





\section{Contained functions} \label{Sec:Contents}

\begin{enumerate}
\item \textbf{MaxScoreCompute.m}  $\rightarrow$  main
\item \textbf{definecAb.m}       $\rightarrow$  defines matrices c,A,b
\item \textbf{definelbub.m}      $\rightarrow$ defines bounds lb,ub and matrices Aeq,beq
\item \textbf{denormalizeEstimates.m}  $\rightarrow$ post-processing
\item \textbf{milp\_cplex.m} $\rightarrow$ calls cplexmilp commercial function
\item \textbf{readXyw.m}  $\rightarrow$ input from files X.txt and ys.txt
\item \textbf{standardizeX.m} $\rightarrow$ pre-processing
\end{enumerate}

\section{Guidelines for usage} \label{Sec:Guidelines}

We present the source code in Matlab in order to compute the MWS (including Max Score) estimator exactly (when technically possible) using MIP following \cite{Florios08}.
There are several options on which MIP solver to use in MATLAB.
Here is one option - for MATLAB $\textregistered$ versions before R2014a:

\begin{itemize}

\item
the cplexmilp.m solver by IBM ILOG CPLEX OPTIMIZATION STUDIO 12.6 $\textregistered$  which is formatted in the milp\_cplex() function we wrote,
and is the state-of-the-art solver for N $\approx$ 500
\end{itemize}

For users of versions after R2014a (including R2014a):

\begin{itemize}
\item 
the intlinprog() solver by MATHWORKS $\textregistered$, can be used for medium scale problems (N=200-400 observations)
\end{itemize}

Typical values for p are p $\in$ \{2,3,4,5,6\}. Often CPLEX can tackle problems up to N  $\approx$ 500 and p $\leq$ 5.
\footnote{It is possible to choose the DGPs in such a way that problems with N $\approx$ 800 and p = 5 can be solved}.


\bibliography{bibfileABCD}


\end{document}



